% !TeX root = mainText.tex
% !TeX root = mainText.tex

\documentclass[a4paper, 12pt]{article}

%\usepackage[margin=0.1in]{geometry}
\usepackage[a4paper, margin=2cm]{geometry}
\usepackage[T2A]{fontenc}
\usepackage[english, russian]{babel}
\usepackage[utf8]{inputenc}
\usepackage{titlesec}
\usepackage{indentfirst}
\usepackage{graphicx}
\graphicspath{ {images/} }
% --- Настройка \section ---
\titleformat{\section}
  {\normalfont\large\bfseries}   % шрифт: обычный, крупный, полужирный
  {\thesection.}                 % нумерация: 1. 2. 3.
  {1em}                          % расстояние между номером и текстом
  {}                             % код до текста (пусто)

\titlespacing*{\section}
  {0pt}{2ex plus 1ex minus .2ex}{1.5ex plus .2ex}

% --- Настройка \subsection ---
\titleformat{\subsection}
  {\normalfont\normalsize\bfseries}
  {\thesubsection.}
  {1em}
  {}

\titlespacing*{\subsection}
  {0pt}{1.5ex plus .5ex minus .2ex}{1ex plus .2ex}


\title{Отчет по плате фотодиода для ПН лазер маяк}
\author{подготовил: Севрюков Дмитрий}
\date{\today}

\begin{document}

\maketitle

\newpage
В данном отчете представлена информация 
о разработанной печатной плате установленной 
на ПН Лазер маяк. Описаны внесенные доработки 
летного экземпляра и приведены результаты измерений
характеристик полученого устройства.

На рисунке \ref{fig:photodiode_PCB} представлены 
фотографии летной версии печатной платы с доработками.

\begin{figure}[h]
    \centering

    \begin{minipage}{0.45\textwidth}
        \centering
        \includegraphics[width=\linewidth]{photodiode_PCB_top.png}
        \\ a)
    \end{minipage}
    \hfill
    \begin{minipage}{0.45\textwidth}
        \centering
        \includegraphics[width=\linewidth]{photodiode_PCB_bottom.png}
        \\ b)
    \end{minipage}

    \caption{Летная версия платы фотодиода: а) вид сверху; б) вид снизу}
    \label{fig:photodiode_PCB}

\end{figure}

\section*{Описание внесенных доработок}

\begin{figure}[h]
    \centering
    \includegraphics[width=\linewidth]{TIA_Circuit.jpg}
    \caption{Схема трансимпедансного усилителя с последующими 
    двумя каскадами усиления, подключенные последовательно}
    \label{fig:TIA_Circuit}
\end{figure}

\begin{itemize}
    \item Конденсатор С5 не установлен, так как он предполагался 
    для проверки некоторых гипотез в последующем.
    \item Резистор R9 и конденсатор С10 также не усановленны, 
    так как компенсация тока смещения и напряжения смещения 
    операционного усилителя была произведена с помошью источника 
    напряяжения на делителе.
    \item Конденсаторы C3, C4, C6, С7, C8, C9 не устанавливаются, 
    так как для фильтрации выходного сигнала при двуполярном
    питании операционного усилителя необходимо подключать 
    конденсаторы на плюс и на минус питания. В противном случае 
    данные конденсаторы выступают в виде ёмкостной нагрузки 
    и схема работает неадекватно.
    \item На резисторы R5 и R7 сверху были напаяны конденсаторы 
    по 1pF для уменьшения коэффициента усиления усилителей 
    для высокочастотных сигналов.
    \item Проводник между U5 (вывод~1) и U4 (вывод~4) был порезан, 
    и между этими выводами был установлен резистор с сопротивлением,
    равным параллельному соединению резисторов в цепи усиления (20~кОм) 
    для компенсации напряжения смещения и тока утечки U4.
    \item Линия между микросхемами U3 (7 вывод) и U4 (4 вывод) 
    была порезана, так как было принято решение подключить 
    два усилителя параллельно, а не последовательно.
    \item Земля от вывода конденсатора С7 была отрезана. 
    На это посадочное место был установлен резистор, 
    компенсирующий смещение напряжения и ток утечки 
    операционного усилителя U3. Номинал резистора выбирается 
    равным параллельному сопротивлению резисторов 
    в цепи усиления (1~кОм).
\end{itemize}

\section*{Оптическая схема измерения фотодиода}
На рисунке \ref{fig:optical_measurement_scheme} представлена
оптическая схема измерения фотодиода.
\begin{figure}[h]
    \centering
    \includegraphics[width=\linewidth]{optical_measurement_scheme.png}
    \caption{Оптическая схема измерения фотодиода}
    \label{fig:optical_measurement_scheme}
\end{figure}

\section*{Результаты измерений передаточных функций}

Результаты измерений передаточной характеристики 
(оптическая мощность - напряжение) первого 
каскада представлены на рисунке \ref{fig:transmission_characteristics_1stage}. 
Таблица измерений находится в файле "Передаточная характеристика 1 каскада.xlsx".
Точка разрыва графика связана с изменением количества 
аттенюаторов (см. оптическая схема измерений)

Результаты измерений передаточной характеристики 
(оптическая мощность - напряжение) второго 
каскада представлены на рисунке \ref{fig:transmission_characteristics_2stage}. 
Таблица измерений находится в файле "Передаточная характеристика 2 каскада.xlsx".

\begin{figure}[h]
    \centering
    \includegraphics[width=\linewidth]{transmission_characteristics_1stage.jpg}
    \caption{Передаточная характеристика первого каскада}
    \label{fig:transmission_characteristics_1stage}
\end{figure}

\begin{figure}[h]
    \centering
    \includegraphics[width=\linewidth]{transmission_characteristics_2stage.jpg}
    \caption{Передаточная характеристика второго каскада}
    \label{fig:transmission_characteristics_2stage}
\end{figure}

\end{document}
